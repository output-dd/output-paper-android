\chapter{Zusammenfassung}\label{ch:zusammenfassung}

Im Verlauf des Sommersemester 2020 wurde ein Framework für das Couchbase-Framework erstellt, welches das bestehende Realm-Framework in der OUTPUT.DD App ersetzen sollte. Während einer Analyse der Apps konnten zahlreiche strukturelle Probleme und Antipatterns in der bestehenden identifiziert. Aus diesem Grund war es Aufgabe dieser Projektarbeit, eine umfängliche Neuimplementation der App inklusive der Integration des Datenbank-Frameworks durchzuführen. Hierbei sollten die in \Cref{ch:einleitung} beschriebenen Probleme beseitigt werden. Gleichzeitig wurden verschiedene Strategien entwickelt die das Wiedereintreten dieser Probleme verhindert. 

\paragraph{Vorgehensweise und Ergebnisse} Hierfür wurden in \Cref{ch:grundlagen} verschiedene Entwurfsmuster beschrieben sowie deren Anwendungsbereiche diskutiert. Durch die Nutzung dieser Design-Patterns wurden signifikante Vorteile für die Separabilität, Funktionalität und Erweiterbarkeit erreicht. Gleichzeitig konnte die Struktur der App analysiert und eine Top-Level-Architektur geschaffen werden, welche die Zerlegung in Teilbereiche der App erlaubt. Durch die Verwendung von statischen Analysetools konnte Codequalität gesteigert und das Eintreten von \enquote{Code Smells} unterbunden werden. Die Neuimplementation der App orientiert sich an bekannten Prinzipien der Usability. Genauso wurde auf die Einhaltung der OUTPUT.DD Corporate Identity geachtet. Gleichzeitig wurde eine Modernisierung der Ansichten durchgeführt und unter Absprache im Projektteam ein optionales dunkles Design eingeführt. Um redundanten Arbeitsaufwand zu minieren wurde auf eine modulare Entwicklung von UI-Komponenten gesetzt, wodurch die Wiederverwendbarkeit von Elementen gewährleistet wurde. Damit konnte die Geschäftslogik der einzelnen Ansichten weiter voneinander separiert werden. Für die Gamification wurden Modelle und Basisalgorithmen entwickelt, um die Errungenschaften weitestgehend unabhängig von den beteiligten UI-Komponenten implementieren zu können. Zum Testen der App wurden konkrete Strategien entwickelt und ein Docker-basiertes Deployment geschaffen. Für die Weiterentwicklung des Projektes wurden neben dieser Dokumentation auch weitere Leitfäden in den Git-Repository der Apps erstellt.

\newpage

\paragraph{Offene Punkte} Die Neuimplementation der OUTPUT.DD-Apps konnte vollständig durchgeführt werden. Neben der Beseitigung von zahlreichen Problemen konnte eine technische und optische Modernisierung der Funktionalitäten umgesetzt werden. Lediglich bei der Integration des Crowd-Monitoring-Framework wurden zahlreiche Probleme identifiziert, die nicht im Rahmen dieser Projektarbeit behoben werden konnten und stattdessen im Fokus zukünftiger Projektarbeiten stehen werden. Im Hinblick auf die OUTPUT.DD 2021 am 8. Juli 2021 werden die Apps weiterhin getestet und Verbesserungen und Optimierungen umgesetzt. Auch die Distribution der Apps und das Deployment auf einem Produktionssystem für OUTPUT.DD 2021 wird noch im Rahmen des Supportprozesses zukünftiger Teil dieser Arbeit sein.

\paragraph{Danksagung}
Diese Projektarbeit hat mir persönlich sehr viel Spaß bereitet. Gleichzeitig konnte ich weitere Kenntnisse im Bereich App-Development sammeln. Besonderen Dank möchte ich Dr. Thomas Springer für die Betreuung dieser Projektarbeit und die zahlreichen Hinweise aussprechen. Außerdem möchte ich meinem guten Freund Philipp Matthes für den sehr konstruktiven Entwicklungsprozess, die Umsetzung der iOS-App und die sehr interessanten Gespräche danken.
